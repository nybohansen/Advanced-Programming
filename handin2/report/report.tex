\documentclass[10pt, a4paper]{article}

\usepackage[utf8]{inputenc}
\usepackage[english]{babel}
\usepackage[T1]{fontenc}

% Package for including code in the document
\usepackage{listings}
\usepackage{color}
\definecolor{blue}{rgb}{0,0,0.8}
\definecolor{green}{rgb}{0,0.5,0}
\definecolor{red}{rgb}{0.5,0,0}
\definecolor{grey}{rgb}{0.5,0.5,0.5}
\lstset{language=Java, 
numbers=left,           % where to put the line-numbers
numberstyle=\tiny,      % size and font of line-numbers
stepnumber=2,           % step between two line-numbers.
numbersep=10pt,         % distance between line-numbers and the code
basicstyle= \ttfamily \footnotesize,  % size and fonts of the code
breaklines=true,        % wrap lines?
showstringspaces=false, % underline spaces within strings
tabsize=2,              % replace tabs with # of spaces
identifierstyle=\color{black},
keywordstyle=\color{blue},
stringstyle=\color{green},
commentstyle=\color{grey},
inputencoding=utf8x,    % text encoding
extendedchars=true,     % exotic characters?
frame=single,             % [none, single, double]
}

% Verbatim envivorment
\usepackage{verbatim} 

% Uncomment if you want to use Palatino as font
\usepackage[sc]{mathpazo}
\linespread{1.05}         % Palatino needs more leading (space between lines)


\title{Advanced Programming \\\small{Assignment 2} - \texttt{MapReduce in Hadoop}}
\author{Mads Ohm Larsen and Kasper Nybo Hansen}

\date{\today}

\begin{document}
	
\maketitle

\section{Introduction} % (fold)
\label{sec:introduction}
For this weeks assignment we had to create two small programs in java, utilizing the Hadoop framework. 

The first program should count the letters in a string, and output each letter and the number of occurrences. We have called this program \texttt{letterCount}.


The second program should output a number $N$ and a word. The word should be chosen with probability $\frac{1}{N}$ where $N$ is the total number of words in the input text. We have called this program \texttt{wordStat}.

Both programs should also be implemented in the Sawzall programming language.
% section introduction (end)

\section{Uniform probability} % (fold)
\label{sec:uniform_probability}
Uniform probability means that all elements have equal probability $1/n$ where $n$ is the number of elements.

We can explain the way our implementation works, by looking at it backwards. 

In the reducer a random integer, \texttt{rndNumber}, in the interval $[0,N]$ is generated. \texttt{rndNumber} and can be thought of as choosing a word by uniform probability since $N$ is the total number of words. \texttt{rndNumber} refers to a section of the total text. In this section, we have already chosen a word with uniform probability, namely $w_i$. This is why the output word is chosen by uniform probability.

% section uniform_probability (end)


\section{Sawzall implementation} % (fold)
\label{sec:sawzall_implementation}
The program letterCount can be implemented in Sawzall as 


\begin{verbatim}
  //letters is defined as a aggregator
  //It sums up the table grouped by key
  letters: table sum[key: char] of int;
  //input is a single letter, that is, the input is split up 
  //into letters before the sawzall program is called
  x: bytes = input;
  //Send 1 to the aggregator along with the key
  emit letters[x] <- 1  
\end{verbatim}

% section sawzall_implementation (end)
               


\section{Test programs}
We have tested both programs on a simple example, consisting of a file containing the string "one on one", and a more advanced example consisting of the provided Shakespeare file.

\subsection{Test of assignment 1} % (fold)
\label{sub:assignment_1}
The first test run on the letterCount program is the simple one consisting of the string "one on one". The output of the program is
\begin{verbatim}
  e	2
  n	3
  o	3
\end{verbatim}
which is the same as if we did a manuel count.

The second test of the letterCount program, the program is run with the provided Shakespeare file as input. The output of the program can bee seen in appendix 1.
% subsection assignment_1 (end)

\subsection{Test of assignment 2} % (fold)
\label{sub:assignment_2}

To test the wordStat program, the program is run with the input $one on one$ three times. The output of the program is

\begin{verbatim}  
  3 on
  3 one
  3 one
  3 one    
  3 on  
\end{verbatim}

Furthermore we have tested this program with the provided shakespeare file as well. The output of the wordStat program when tested with the shakespeare file is

\begin{verbatim}
  1082023 Porter
  1082023 weak
  1082023 is
  1082023 PAGE
  1082023 him
\end{verbatim}

% subsection assignment_2 (end)

\newpage
\section*{Appendix 1} % (fold)
\label{sec:appendix_1}
Output of character count program when run on shakespeare
\begin{verbatim}
  a	307310
  b	63387
  c	92297
  d	158302
  e	477296
  f	85216
  g	72708
  h	250992
  i	269855
  j	4947
  k	38057
  l	179484
  m	116224
  n	258597
  o	329097
  p	60665
  q	4095
  r	250672
  s	263343
  t	349494
  u	136832
  v	41106
  w	94026
  x	5561
  y	98341
  z	1858
\end{verbatim}

% section appendix_1 (end)

\newpage
\section{Source code} % (fold)

\subsection{LetterCountMapper} % (fold)
\label{sub:lettercountmapper}
\begin{lstlisting}
  package letter;

  import java.io.IOException;
  import java.util.StringTokenizer;
  import org.apache.hadoop.io.LongWritable;
  import org.apache.hadoop.io.Text;
  import org.apache.hadoop.mapred.FileSplit;
  import org.apache.hadoop.mapred.MapReduceBase;
  import org.apache.hadoop.mapred.Mapper;
  import org.apache.hadoop.mapred.OutputCollector;
  import org.apache.hadoop.mapred.Reporter;
  import org.apache.hadoop.mapred.JobConf;

  public class LetterCountMapper extends MapReduceBase
      implements Mapper<LongWritable, Text, Text, Text> {

    public LetterCountMapper() { }

    public void map(LongWritable key, Text value, OutputCollector<Text, Text> output,
        Reporter reporter) throws IOException {

  	    //Split string into array of chars. Remove all characters that are not in [a-z]  
  	    char[] letterArray = value.toString().toLowerCase().replaceAll("[^a-z]","").toCharArray();

  	    //For each letter send it to the reducer
  	    for(int i=0; i<letterArray.length; i++){
  	    	output.collect( new Text(String.valueOf(letterArray[i])), new Text("1"));
  	    }
  	  }
  }
\end{lstlisting}

% subsection lettercountmapper (end)

\newpage
\subsection{LetterCountReducer} % (fold)
\label{sub:lettercountreducer}

\begin{lstlisting}
package letter;

import java.io.IOException;
import java.util.Iterator;
import org.apache.hadoop.io.Text;
import org.apache.hadoop.mapred.MapReduceBase;
import org.apache.hadoop.mapred.OutputCollector;
import org.apache.hadoop.mapred.Reducer;
import org.apache.hadoop.mapred.Reporter;
import org.apache.hadoop.mapred.JobConf;


public class LetterCountReducer extends MapReduceBase
    implements Reducer<Text, Text, Text, Text> {

  public LetterCountReducer() { }

  public void reduce(Text key, Iterator<Text> values,
      OutputCollector<Text, Text> output, Reporter reporter) throws IOException {
	  	int sum = 0;
	    while (values.hasNext()) {
	    	//Sum is total number of occurrences of the letter in key
	    	sum += Integer.parseInt(values.next().toString());
	    }
	    //Output the result
	    output.collect(key, new Text(Integer.toString(sum)));
	  }
}
\end{lstlisting}

% subsection lettercountreducer (end)


\subsection{WordStatMapper} % (fold)
\label{sub:wordstatmapper}
\begin{lstlisting}
package wordStat;

import java.io.IOException;
import java.util.StringTokenizer;

import org.apache.hadoop.io.LongWritable;
import org.apache.hadoop.io.Text;
import org.apache.hadoop.mapred.FileSplit;
import org.apache.hadoop.mapred.MapReduceBase;
import org.apache.hadoop.mapred.Mapper;
import org.apache.hadoop.mapred.OutputCollector;
import org.apache.hadoop.mapred.Reporter;
import org.apache.hadoop.mapred.JobConf;

import java.util.Random;

public class WordStatMapper extends MapReduceBase
    implements Mapper<LongWritable, Text, Text, Text> {

  public WordStatMapper() { }

  public void map(LongWritable key, Text value, OutputCollector<Text, Text> output,
      Reporter reporter) throws IOException {
	 
	  //Split string into word array
	  String wordArray[] = value.toString().split("\\s+");
	  
	  //number of words
	  int numberOfWords = wordArray.length;
	  
	  //Pick one random word from the word array. This works because the word array
	  //contains all the words including any duplicates. E.x. "one on one" becomes ["one", "on", "one"]
	  //when selecting a random word in this array, the probability to hit "one" is still 2/3.
	  String pickedWord = "";
	  
	  if(numberOfWords>0){
		  //If we are not operating on a blank line, i.e. a line which only contains blank chars.
		  Random rndGenerator = new Random( key.get() + java.util.Calendar.getInstance().getTimeInMillis());
		  pickedWord = wordArray[rndGenerator.nextInt(numberOfWords)];
	  }
	 
	  //Send it to the reducer
	  output.collect( new Text("key"), new Text(pickedWord +" " + Integer.toString(numberOfWords)));
	 
  }
}  
\end{lstlisting}
% subsection wordstatmapper (end)


\subsection{WordStatReducer} % (fold)
\label{sub:wordstatreducer}

\begin{lstlisting}
package wordStat;

import java.io.IOException;
import java.util.ArrayList;
import java.util.Iterator;
import java.util.List;
import java.util.Random;

import org.apache.hadoop.io.Text;
import org.apache.hadoop.mapred.MapReduceBase;
import org.apache.hadoop.mapred.OutputCollector;
import org.apache.hadoop.mapred.Reducer;
import org.apache.hadoop.mapred.Reporter;
import org.apache.hadoop.mapred.JobConf;

import java.util.Random;

public class WordStatReducer extends MapReduceBase
    implements Reducer<Text, Text, Text, Text> {

  public WordStatReducer() { }

  public void reduce(Text key, Iterator<Text> values,
      OutputCollector<Text, Text> output, Reporter reporter) throws IOException {
	  
	  int N = 0;
	  int n_i = 0;
	  String w_i;
	  //Array of all the words
	  List<String> wordList = new ArrayList<String>();
	  //Array of the sum 
	  List<Integer> sumList = new ArrayList<Integer>(); 
	  
	  while (values.hasNext()) {
		  //Create the array Payload containin the word and the number of words
		  String payload[] = values.next().toString().split(" ");
		  //Extract the pair from the string
		  w_i = payload[0];
		  n_i = Integer.parseInt(payload[1]);  
		  //If the mapper processed a string containing no chars, we want
		  //n_i is 0. 
		  if(n_i>0){
			  N += n_i;
			  wordList.add(w_i);
			  //Add the accumulated sum
			  sumList.add(N);
		  }
	  }

	  //Seed random generator as explained in the assignment
	  Random rndGenerator = new Random(java.util.Calendar.getInstance().getTimeInMillis()); 
	  //Select random number in the interval [0,N] 
	  int rndNumber = rndGenerator.nextInt(N);
	  
	  //Find the place in the word array where the picked word is
	  int i = 0;
	  while(sumList.get(i)<rndNumber){
		i++;    
	  }
	  
	  //Store the picked word
	  String pickedWord = wordList.get(i);
	  //Send it to the output
	  output.collect(key, new Text(Integer.toString(N) + " " + pickedWord));  
  }
}  
\end{lstlisting}
% subsection wordstatreducer (end)


% \lstinputlisting{../Assignment1/src/letter/LetterCountMapper.java}


% section source_code (end)



\end{document}
