\documentclass[10pt, a4paper]{article}

\usepackage[utf8]{inputenc}
\usepackage[english]{babel}
\usepackage[T1]{fontenc}

% Package for including code in the document
\usepackage{listings}
\usepackage{color}
\definecolor{blue}{rgb}{0,0,0.8}
\definecolor{green}{rgb}{0,0.5,0}
\definecolor{red}{rgb}{0.5,0,0}
\definecolor{grey}{rgb}{0.5,0.5,0.5}
\lstset{language=Erlang, 
numbers=left,           % where to put the line-numbers
numberstyle=\tiny,      % size and font of line-numbers
stepnumber=2,           % step between two line-numbers.
numbersep=10pt,         % distance between line-numbers and the code
basicstyle= \ttfamily \footnotesize,  % size and fonts of the code
breaklines=true,        % wrap lines?
showstringspaces=false, % underline spaces within strings
tabsize=2,              % replace tabs with # of spaces
identifierstyle=\color{black},
keywordstyle=\color{blue},
stringstyle=\color{green},
commentstyle=\color{grey},
inputencoding=utf8x,    % text encoding
extendedchars=true,     % exotic characters?
frame=single,             % [none, single, double]
}

% Verbatim envivorment
\usepackage{verbatim} 

% Uncomment if you want to use Palatino as font
\usepackage[sc]{mathpazo}
\linespread{1.05}         % Palatino needs more leading (space between lines)

\title{Advanced Programming \\\small{Assignment 3} - \texttt{Mayo Peer-to-Peer}}
\author{Kasper Nybo Hansen and Mads Ohm Larsen}

\date{October 22, 2010}

\begin{document}
	
\maketitle

\section{Introduction} % (fold)
\label{sec:introduction}
This week we had to create a concurrent peer-to-peer (P2P) phonebook program in Erlang.
The implementation should be an API that allows the following operations:

\begin{itemize}
	\item Add a contact
	\item List all contacts
	\item Update a contact, given the name
	\item Delete a contact, given the name
	\item Find a contact, given the name
\end{itemize}

The individual peers should also be able to join the network and share the contacts.
% section introduction (end)

\section{The Mayo Routing Algorithm} % (fold)
\label{sec:the_mayo_routing_algorithm}
We are using the Mayo Routing Algorithm to maintain our contacts in our phonebook.
Each peer have got a GUID, which is a 128-bit MD5-hash of their name.
Each peer have got the contacts which has a higher MD5-hash than the peer, up til the next peer.
The left most peer have also got the contacts below.

When doing this there were some thing we had to take into mind.
What happens if $P_1$ and $P_2$ makes a network together, and shares some contacts, and $P_3$ and $P_4$ makes a networks and shares some contacts, and $P_3$ joins $P_1$?
In our implementation this results in the $P_3$/$P_4$ network all call the \texttt{join}-method on $P_1$, and are routed into the correct place.
The contacts are also all given to $P_1$, which finds the correct peer for the contact, and gives it to that peer (Might be $P_3$ or $P_4$, but could equally likely be $P_2$).

\section{Implementation} % (fold)
\label{sec:implementation}
To the outside world, only the following methods are exported, and the outside world is therefore only allowed to call these:
\begin{itemize}
	\item \texttt{start/1}
	\item \texttt{add/2}
	\item \texttt{list\_all/1}
	\item \texttt{update/2}
	\item \texttt{delete/2}
	\item \texttt{lookup/2}
	\item \texttt{join/2}
	\item And two test methods
	\begin{itemize}
		\item \texttt{jointest/0}
		\item \texttt{test/0}
	\end{itemize}
\end{itemize}
% section implementation (end)

They are pretty much selfexplanatory, but we'll discuss one of our utility methods, that all the methods concerning a specific contact uses, namely the \texttt{findCorrectPeer/2} method.
This method takes a Pid and a GUID, and finds the correct peer for that GUID, that is, the one which has the highest GUID below this.
This is done by checking with the neighbours, firstly if they are active (that is, that the peer we are calling is not the left or right most), secondly if we should move further left or right, or if we have got the correct peer.

\begin{lstlisting}
findCorrectPeer(InNetPeer, GUID) ->
    case GUID < getGUID(InNetPeer) of
        true -> LeftPeer = getNeighbour(InNetPeer, left),
                case isActive(LeftPeer) of
                    true  -> findCorrectPeer(LeftPeer, GUID);
                    false -> InNetPeer
                end;
        false -> RightPeer = getNeighbour(InNetPeer, right),
                 case isActive(RightPeer) of
                     true  -> case getGUID(RightPeer) < GUID of
                                  true  -> findCorrectPeer(RightPeer, GUID);
                                  false -> InNetPeer
                              end;
                     false -> InNetPeer
                 end
    end.
\end{lstlisting}

This function is called prior to sending the \texttt{rpc}-call to the Pid, so we actually change the Pid, if it isn't correct.
An example of this is the \texttt{add/2}-method.

\begin{lstlisting}
add(Pid, Contact) ->
    {Name, _, _} = Contact,
    Peer = findCorrectPeer(Pid, erlang:md5(Name)),
    rpc(Peer, {add, Contact}).	
\end{lstlisting}

Our \texttt{join/2}-method also utilizes this method, to find the correct peer to join.
The \texttt{join/2}-method also makes all the other peers in that network join the same peer in the new network.
Doing so, we are able to join two networks together, sharing all the contacts.
However, we are not doing any robustness check, so if the user were to make a peer join a network, that it was already part of, the system would hang.
% section the_mayo_routing_algorithm (end)

\section{Testing} % (fold)
\label{sec:testing}

The two test methods, \texttt{jointest/0} and \texttt{test/0}, we mentioned earlier, tests various parts of our implementation.

\begin{lstlisting}
jointest() -> P1 = start("P1"),
              P2 = start("P2"),
              P3 = start("P3"),
              P4 = start("P4"),
              join(P2, P1),
              join(P4, P3),
              join(P3, P2),
              printChain(P4).
%% Prints "end - P2 - P1 - P4 - P3 - end"

test() -> 
  P1 = start("P1"),
  add(P1, {"Donald Duck", "Duckburg", "1313-13-1313"}),
  add(P1, {"Huey Duck", "Duckburg, at Donalds", "555-HUEY-DUCK"}),
  P2 = start("P2"),
  add(P2, {"Dewey Duck", "Duckburg, at Donalds", "555-DEWE-DUCK"}),
  add(P2, {"Louie Duck", "Duckburg, at Donalds", "555-LOUI-DUCK"}),
  join(P2, P1),
  P3 = start("P3"),
  add(P3, {"Scrooge McDuck", "Duckburg, at his money bin", "1"}),
  P4 = start("P4"),
  add(P4, {"Daisy Duck", "Unknown", "12345678"}),
  join(P3, P4),
  join(P4, P1),
  delete(P3, "Daisy Duck"),
  list_all(P2).
%% {ok,[{"Louie Duck","Duckburg, at Donalds","555-LOUI-DUCK"},
%%      {"Donald Duck","Duckburg","1313-13-1313"},
%%      {"Dewey Duck","Duckburg, at Donalds","555-DEWE-DUCK"},
%%      {"Huey Duck","Duckburg, at Donalds","555-HUEY-DUCK"},
%%      {"Scrooge McDuck","Duckburg, at his money bin","1"}]}
\end{lstlisting}
% section testing (end)

\section{Assesment} % (fold)
\label{sec:assesment}

As mentioned earlier, our implementation do not check for robustness, that is, we trust the user to not do weird stuff, like making a peer join itself or making a peer join a network it is already part of.
We also do not consider the case that a peer breaks down or leaves the network unexpected.
Other than these weird cases, our implementation should be able to handle all the calls required in creating a peer-to-peer phone book.

% section assesment (end)


\end{document}