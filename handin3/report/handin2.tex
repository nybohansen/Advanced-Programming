\documentclass[10pt, a4paper]{article}

\usepackage[utf8]{inputenc}
\usepackage[english]{babel}
\usepackage[T1]{fontenc}

% Package for including code in the document
\usepackage{listings}
\usepackage{color}
\definecolor{blue}{rgb}{0,0,0.8}
\definecolor{green}{rgb}{0,0.5,0}
\definecolor{red}{rgb}{0.5,0,0}
\definecolor{grey}{rgb}{0.5,0.5,0.5}
\lstset{language=Erlang, 
numbers=left,           % where to put the line-numbers
numberstyle=\tiny,      % size and font of line-numbers
stepnumber=2,           % step between two line-numbers.
numbersep=10pt,         % distance between line-numbers and the code
basicstyle= \ttfamily \footnotesize,  % size and fonts of the code
breaklines=true,        % wrap lines?
showstringspaces=false, % underline spaces within strings
tabsize=2,              % replace tabs with # of spaces
identifierstyle=\color{black},
keywordstyle=\color{blue},
stringstyle=\color{green},
commentstyle=\color{grey},
inputencoding=utf8x,    % text encoding
extendedchars=true,     % exotic characters?
frame=single,             % [none, single, double]
}

% Verbatim envivorment
\usepackage{verbatim} 

% Uncomment if you want to use Palatino as font
\usepackage[sc]{mathpazo}
\linespread{1.05}         % Palatino needs more leading (space between lines)

\title{Advanced Programming \\\small{Assignment 3} - \texttt{Mayo Peer-to-Peer}}
\author{Kasper Nybo Hansen and Mads Ohm Larsen}

\date{\today}

\begin{document}
	
\maketitle

\section{Introduction} % (fold)
\label{sec:introduction}
This week we had to create a concurrent peer-to-peer (P2P) phonebook program in Erlang.
The implementation should be an API that allows the following operations:

\begin{itemize}
	\item Add a contact
	\item List all contacts
	\item Update a contact, given the name
	\item Delete a contact, given the name
	\item Find a contact, given the name
\end{itemize}

The individual peers should also be able to join the network and share the contacts.
% section introduction (end)

\section{The Mayo Routing Algorithm} % (fold)
\label{sec:the_mayo_routing_algorithm}
We are using the Mayo Routing Algorithm to maintain our contacts in our phonebook.
Each peer have got a GUID, which is a 128-bit MD5-hash of their name.
Each peer have got the contacts which has a higher MD5-hash than the peer, up til the next peer.
The left most peer have also got the contacts below.

When doing this there were some thing we had to take into mind.
What happens if $P_1$ and $P_2$ makes a network together, and shares some contacts, and $P_3$ and $P_4$ makes a networks and shares some contacts, and $P_3$ joins $P_1$?
In our implementation this results in the $P_3$/$P_4$ network all call the \texttt{join}-method on $P_1$, and are routed into the correct place.
The contacts are also all given to $P_1$, which finds the correct peer for the contact, and gives it to that peer (Might be $P_3$ or $P_4$, but could equally likely be $P_2$).


% section the_mayo_routing_algorithm (end)
\end{document}